
\chapter*{科研成果发表}

论文发表

1.Chao Yang, Lei Xie and Xiangzeng Zhou,"Unsupervised broadcast news story segmentation using distance dependent Chinese restaurant process",IEEE International Conference on Acoustics, Speech, and Signal Processing,Florence, Italy,2014(EI索引,语音研究顶级国际会议)

科研工作:

1.国家自然科学基金面上项目《基于DBN协同建模的中文及跨语种语音结构事件检测研究》(61175018)

\cleardoublepage

\chapter*{致谢}
随着毕业论文的尘埃落定,我的校园生活也接近尾声。借此机会,对所有帮助过我的师长、亲人、同学、朋友表达我真挚的感谢。

首先,我要衷心的感谢我的导师谢磊教授。他有着严谨的治学理念和精益求精的态度,将大部分时间都奉献给了科研事业和对学生的指导。正是在他一步步耐心引导下,我才接触到这一有趣且富有挑战性的研究领域并逐渐掌握了正确的研究思路和方法。本论文的完成过程是极为不易的,在论文选题,文献阅读,理论研究,实验分析,论文撰写每个阶段,谢老师都在不遗余力的指导和鼓励着我,可以说这篇论文里处处都有他的心血。在此,谨向谢老师表达我我最衷心的感谢,您的教诲我会永远铭记。

感谢蒋冬梅教授和付中华教授。两位老师踏实的研究态度、渊博的学科知识、和乐观豁达的生活态度,一直深深地感染着我并将使我受益终生。

感谢西北大学的冯筠老师认真审阅了全篇论文,并提出了宝贵的意见。

感谢我的师兄师姐:田霄海,赵文淮,芦咪咪,郑李磊,史倩,吴杰,何颖,周详增,赵亚丽,孙乃才,李冰锋。他们毫无保留的将自己的学习经验和方法与我分享,使我在研究领域得到了更快的进步。尤其要感谢周详增师兄,他仔细阅读了我的论文并提出了宝贵的修改意见。

感谢教研室的同学:牛建伟,路晓明,高新远,原帅,李龙飞,陈爱华,唐玲,刘洋。这两年多来大家相互鼓励,相互帮助,一起经历了科研和生活中的辛酸与甘甜,是我人生中永远难忘的回忆。感谢许军海同学,协助我完成了本文部分实验内容。感谢于佳和陈宏杰两位师弟,他们为我的论文修改提出了宝贵的意见。

感谢秦巧玲同学一直以来对我的关心、照顾和理解。

最后,要感谢我的父母这些年对我含辛茹苦的养育和教导,他们一直默默地为我奉献着最无私的爱,我希望能尽早报答这一切。
\cleardoublepage
\includepdfmerge{./copyright,1-1}