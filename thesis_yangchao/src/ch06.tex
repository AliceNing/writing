\chapter{总结与展望}
本文对狄利克雷过程相关的方法进行了详细的研究,包括将其应用在经典的概率模型如混合模型,超混合模型和时序模型上的方法,以及相关的基于采样的推断算法。通过对经典的参数模型加入狄利克雷过程先验,可以有效地解决人工模型选择的繁杂任务。

本文主要研究了狄利克雷过程在时序模型上的具体应用任务,利用依赖于距离的中国餐馆过程对故事分割任务进行建模,并与多个主流的基线系统比较,实验表明其具有非常好的结果。而另一种对时序数据进行建模的思路是建立马尔科夫模型,本文通过引入分层狄利克雷过程先验,对隐马尔科夫模型进行非参数化,并将其应用在类音素发现的任务上,得到了较好的实验结果。

文中主要使用的是基于gibbs采样的方法,这对于大数据问题会存在一些效率问题。面对互联网上海量的数据,需要进一步研究更加有效的算法。目前,关于狄利克雷过程相关的一些变分方法的研究非常重要,也是相关学术研究中的热点之一,所以下一步考虑利用变分方法来加快模型的求解速度。另一方面,对于许多任务,数据是在不断更新增加的,所以在线算法也显得非常重要,这也是下一步需要研究的重点。

对于语料中的多篇文档,实际上本身就有一个边界信息,但是本文在建模时并没有利用上。在故事分割任务中,由于没有去建模出每个段落之间共享特征这个性质,所以对每个新闻语料单独处理是合理的。但是,对于音素发现的任务,每篇文档(在这个语料里即每一句话)之间其实是条件独立的,简单的将所有文档拼接成一个连续的序列,就损失了一些天然的性质。为了建模这一条件独立性质,可以利用贝塔过程\cite{fox2009sharing}相关的理论进行建模,这也是后续研究的一个重要内容。